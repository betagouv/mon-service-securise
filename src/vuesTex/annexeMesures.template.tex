\documentclass[9pt, a4paper]{article}
\usepackage[french]{babel}
\usepackage[T1]{fontenc}


\usepackage{array}
\usepackage[a4paper, margin=10mm]{geometry}

\usepackage[scaled]{helvet}
\renewcommand*\familydefault{\sfdefault}

\usepackage[inkscape=off,inkscapepath=svgpath]{svg}

\usepackage{multicol}
\usepackage{nopageno}

\usepackage{tcolorbox}

\usepackage{xcolor}
\definecolor{bleu}{rgb}{0.03, 0.25, 0.42}
\definecolor{gris}{rgb}{0.37, 0.37, 0.37}
\definecolor{gris_clair}{rgb}{0.96, 0.96, 0.96}
\definecolor{lisere}{rgb}{0.71, 0.76, 0.82}

\setlength\parindent{0cm}

\begin{document}
  \textbf{MESURES DE SÉCURITÉ DÉTAILLÉES}

  \textcolor{gris}{Toutes les mesures indispensables *, recommandées et créées sont classées selon
  leur statut de mise en œuvre et par catégorie.}

  \vskip 0.5cm

  \textbf{En cours}

  \begin{tcolorbox}[colback=white, colframe=lisere, boxrule=1px]
    \textcolor{bleu}{__= donnees.categorie __}
    \begin{itemize}
      \item * Limiter et connaître les interconnexions du service avec d'autres systèmes

        \textcolor{gris}{Le service est complètement autonome et n'est connecté à aucun autre
        service.}
      \item * Disposer d'une liste à jour des équipements et des logiciels contribuant au
        fonctionnement du service numérique
      \item Identifier les données les plus sensibles à protéger
    \end{itemize}
  \end{tcolorbox}

  \vskip 1cm

  \textcolor{bleu}{\textbf{MonServiceSécurisé}}, service proposé par l'ANSSI.
\end{document}
